\documentclass[11pt]{article}
\usepackage[paperwidth=8.5in, paperheight=11in]{geometry}

\usepackage{../tjimo}
%\usepackage[pdftex]{graphicx}
\usepackage{asymptote}

%\begin{comment}
\def\answer{\comment}
\def\solution{\comment}
\def\solutionone{\comment}
\def\solutiontwo{\comment}
%\end{comment}

\newcommand{\sevenpoints}{Time limit: 30 minutes}
\newcommand{\righthead}{\fdbox{Round}{Team}}

\begin{document}

\begin{problem}
If the prime factorization of $60^6$ is  $2^a \cdot 3^b \cdot 5^c$, where $a,b,c$ are positive integers, find $a+b+c$.
\end{problem}
\begin{answer}
$\boxed{24}$
\end{answer}
\begin{solution}
Notice $60^6 = (2^2 \cdot 3 \cdot 5)^6 = 2^{12} \cdot 3^6 \cdot 5^6$. Thus our answer is $12 + 6 + 6 = \boxed{24}$.
\end{solution}


\begin{problem}
A bag contains marbles of three colors: 10 red marbles, p green marbles, and 55 blue marbles. The probability of randomly selecting a green marble from the bag is $\dfrac{p}{90}$. Find the probability of selecting a blue marble.
\end{problem}
\begin{answer}
$\boxed{\dfrac{11}{18}}$
\end{answer}
\begin{solution}
 Notice the probability of selecting a green marble from the bag is $\dfrac{p}{55 + 10 + p} = \dfrac{p}{65 + p}$. If this is equal to $\dfrac{p}{90}$, then $65 + p = 90$ and $p = 25$, so the probability of selecting a blue marble is $\dfrac{55}{10 + 25 + 55} = \dfrac{55}{90} = \boxed{\dfrac{11}{18}}$.
\end{solution}

\begin{problem}
The area of equilateral triangle $ABC$ (AB = BC = CA) is $300$. $O$ is the center of the triangle, as shown below. 
\begin{figure}[H]
\begin{center}
\begin{asy}
import graph;
size(80);
draw((0,0)--(1,1.732)--(2,0)--(0,0));
draw((0,0)--(1, 0.5897)--(2,0));
defaultpen(fontsize(10pt));
label("A", (0, 0), SW);
label("B", (1, 1.732), N);
label("C", (2, 0), SE);
label("O", (1, 0.5773), N);
dot((1, 0.5773));
\end{asy}
\end{center}
\end{figure}
Find the area of triangle $AOC$. 
\end{problem}
\begin{answer}
$\boxed{100}$
\end{answer}
\begin{solution}
Notice by symmetry $[AOB] = [BOC] = [COA]$ and they sum to $300$ so our answer is $\boxed{100}$.
\end{solution}


\begin{problem}
Given that $x,y$ are positive real numbers satisfying $x + \dfrac{1}{y} = 5$ and $y + \dfrac{1}{x} = 7$, find $\dfrac{x}{y}$.
\end{problem}
\begin{answer}
$\boxed{\dfrac{5}{7}}$
\end{answer}
\begin{solution}
 We can clear denominators in both equations to obtain $xy + 1 = 5y$ and $xy + 1 = 7x$. Thus, $5y = 7x$ and we see $\dfrac{x}{y} = \boxed{\dfrac{5}{7}}$.
\end{solution}


\begin{problem}
Let right triangle ABC have $AB = 6$, $BC = 8$, and $CA = 10$. Let $X$ be the midpoint of $AB$, let $Y$ be the midpoint of $BC$, and let $Z$ be the midpoint of $CA$. Find the area of $XYZ$. 
\end{problem}
\begin{answer}
$\boxed{6}$ 
\end{answer}
\begin{solution}
The area of the middle triangle is $\dfrac{1}{4}$ the area of the outer triangle, so our answer is $\dfrac{1}{4} \cdot \dfrac{1}{2} \cdot 6 \cdot 8 = \boxed{6}$. 
\end{solution}

\begin{problem}
Alex the ant is at point A on the below grid.
\begin{figure}[H]
\begin{center}
\begin{asy}
size(70);
defaultpen(fontsize(8pt));
draw((0,0)--(0,2)--(2,2)--(2,0)--(0,0));
draw((2,2)--(2,4)--(4,4)--(4,2)--(2,2));
draw((1,0)--(1,2));
draw((0,1)--(2,1));
draw((3,2)--(3,4));
draw((2,3)--(4,3));
label("A", (0, 0), SW);
label("B", (4, 4), NE);
\end{asy}
\end{center}
\end{figure}
 If he can only move up or to the right on the grid, find the number of distinct paths Alex can take to go from point A to B.
\end{problem}
 
\begin{answer}
$\boxed{36}$ (paths)
\end{answer}
\begin{solution}
Label the middle point as $C$. We can count $6$ paths to go from $A$ to $C$. (Alternatively, we notice that to go from $A$ to $C$, we must move up twice and move to the right twice. Then there are $\dbinom{4}{2}$ ways to choose the order in which to make the moves) Then similarly there are $6$ paths to go from $C$ to $B$. Thus, there are $6 \cdot 6 = \boxed{36}$ total paths to go from $A$ to $B$.
\end{solution}

\begin{problem}
If $x,y$ are distinct positive integers such that \[ x^2 + 2xy = 40 \] \[y^2 + 2xy = 33\] find $x+y$.
\end{problem}
\begin{answer}
\boxed{7}
\end{answer}
\begin{solution}
Subtracting the equations gives $x^2 - y^2 = 7$ or $(x-y)(x+y) = 7$. We see $x+y = \boxed{7}$. 
\end{solution}

\begin{problem}
Suppose square $ABCD$ has side length of $10$ and circle $O$ is inscribed in the square, as shown below.
\begin{figure}[H]
\begin{center}
\begin{asy}
import graph;
size(80);
draw((0,0)--(0,2)--(2,2)--(2,0)--(0,0));
draw(Circle((1,1),1));
draw((0, 0) -- (2, 2));
defaultpen(fontsize(10pt));
label("A", (0, 2), NW);
label("B", (2, 2), NE);
label("C", (2, 0), SE);
label("D", (0, 0), SW);
label("O", (1, 1), S);
label("X", (1.707, 1.707), S);
dot((1, 1));
\end{asy}
\end{center}
\end{figure} Segment $BO$ intersects the circle again at $X$. Find the length of $BX$.
\end{problem}
\begin{answer}
$\boxed{5\sqrt{2} - 5}$
\end{answer}
\begin{solution}
Let the other intersection of $BD$ and circle $O$ be $Y$. Notice by symmetry, $BX = DY$. Also, $BX + XY + YD = BD = 10\sqrt{2}$. We know $XY$ is the diameter of circle $O$, so $XY = 10$. Thus $BX + YD = 10\sqrt{2} - 10$, so $BX = \boxed{5\sqrt{2} - 5}$.
\end{solution}


\begin{problem}
Joe lists all the three digit positive integers on a sheet of paper. However, he skips all the integers that are divisible by three and all the integers that are divisible by five. Determine how many integers Joe counts.
\end{problem}
\begin{answer}
\boxed{480} (integers)
\end{answer}
\begin{solution}
 We use the Principle of Inclusion-Exclusion (PIE). There are $900$ total three-digit positive integers. We subtract off the $300$ multiples of three and $180$ multiples of five, and add the $60$ multiples of $15$ to obtain $900 - 300 - 180 + 60 = \boxed{480}$.
\end{solution}

\begin{problem}
Compute the number of positive integers $1 \le x \le 143$ such that $x^2 + x^3$ is a perfect square.
\end{problem}
\begin{answer}
$\boxed{11}$ (integers)
\end{answer}
\begin{solution}
If $x^2 + x^3 = x^2(x+1)$ is the square of an integer, then $(x+1)$ must be the square of an integer. We also know $2 \le x+1 \le 144$. Thus $x+1 \in \{2^2, 3^2, \cdots, 12^2\}$, so there are $\boxed{11}$ possible values of $x$.
\end{solution}


\end{document}