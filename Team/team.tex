\documentclass[11pt]{article}
\usepackage[paperwidth=8.5in, paperheight=11in]{geometry}

\usepackage{../tjimo}
%\usepackage[pdftex]{graphicx}
\usepackage{asymptote}

%\begin{comment}
\def\answer{\comment}
\def\solution{\comment}
\def\solutionone{\comment}
\def\solutiontwo{\comment}
%\end{comment}

\newcommand{\sevenpoints}{Time limit: 30 minutes}
\newcommand{\righthead}{\fdbox{Round}{Team}}

\begin{document}

\begin{problem}
If the prime factorization of $60^6$ is  $2^a \cdot 3^b \cdot 5^c$, where $a,b,c$ are positive integers, find $a+b+c$.
\end{problem}
\begin{answer}
33
\end{answer}
\begin{solution}
Notice $60^6 = (2^2 \cdot 3 \cdot 5)^6 = 2^{12} \cdot 3^6 \cdot 5^6$. Thus our answer is $12 + 6 + 6 = 24$.
\end{solution}


\begin{problem}
A bag contains marbles of three colors: 10 red marbles, p green marbles, and 55 blue marbles. The probability of randomly selecting a green marble from the bag is $\dfrac{p}{90}$. Find the probability of selecting a blue marble.
\end{problem}
\begin{answer}
$\dfrac{11}{18}$
\end{answer}
\begin{solution}
 Notice the probability of selecting a green marble from the bag is $\dfrac{p}{55 + 10 + p} = \dfrac{p}{65 + p}$. If this is equal to $\dfrac{p}{90}$, then $65 + p = 90$ and $p = 25$, so the probability of selecting a blue marble is $\dfrac{55}{10 + 25 + 55} = \dfrac{55}{90} = \dfrac{11}{18}$.
\end{solution}


\begin{problem}
Joe lists all the three digit positive integers on a sheet of paper, starting with $100$ and ending with $999$. However, he skips all the integers that are divisible by three and all the integers that are divisible by five. How many positive integers does he count?
\end{problem}
\begin{answer}
\boxed{480}
\end{answer}
\begin{solution}
 We use the Principle of Inclusion-Exclusion (PIE). There are $900$ total positive integers. We subtract off the $300$ multiples of three and $180$ multiples of five, and add the $60$ multiples of $15$ to obtain $900 - 300 - 180 + 60 = \boxed{480}$.
\end{solution}


\begin{problem}
Given that $x,y$ are positive real numbers satisfying $x + \dfrac{1}{y} = 5$ and $y + \dfrac{1}{x} = 7$, find $xy + \dfrac{1}{xy}$.
\end{problem}
\begin{answer}
$33$
\end{answer}
\begin{solution}
 Notice $\left(x + \dfrac{1}{y}\right)\left(y + \dfrac{1}{x}\right) = xy + 2 + \dfrac{1}{xy}$ which is also equal to $5 \cdot 7 = 35$, so $xy + \dfrac{1}{xy} = 35-2 = \boxed{33}$.
\end{solution}


\begin{problem}
For how many positive integers $1 \le x \le 143$ is $x^2 + x^3$ the square of an integer? 
\end{problem}
\begin{answer}
$\boxed{11}$
\end{answer}
\begin{solution}
If $x^2 + x^3 = x^2(x+1)$ is the square of an integer, then $(x+1)$ must be the square of an integer. We also know $2 \le x+1 \le 144$. Thus $x+1 \in \{2^2, 3^2, \cdots, 12^2\}$, so there are $11$ possible values of $x$.
\end{solution}


\begin{problem}
Let triangle ABC have $AB = 7$, $BC = 24$, and $CA = 25$. Let $X$ be the midpoint of $AB$, let $Y$ be the midpoint of $BC$, and let $Z$ be the midpoint of $CA$. Find the area of $XYZ$. 
\end{problem}
\begin{answer}
$28$. 
\end{answer}
\begin{solution}
The area of the middle triangle is $\dfrac{1}{4}$ the area of the outer triangle, so our answer is $\dfrac{1}{4} \cdot \dfrac{1}{2} \cdot 7 \cdot 24 = 21$. 
\end{solution}


\begin{problem}
How many distinct ways are there to place the positive integers $1, 2, 3, 4, 5$ into the squares below such that every integer is used exactly once, each square contains exactly one integer, the horizontal row of three integers is in either increasing or decreasing order, and the vertical column of three integers is in either increasing or decreasing order?
\begin{figure}[H]
	\begin{center}
	\begin{asy}
	size(40);
	draw((0,0)--(0,1)--(1,1)--(1,2)--(2,2)--(2,1)--(3,1)--(3,0)--(2,0)--(2,-1)--(1,-1)--(1,0)--(0,0));
	draw((1,1)--(1,0)--(2,0)--(2,1)--(1,1));
	\end{asy}
	\end{center}
\end{figure}
\end{problem}

\begin{answer}
16
\end{answer}
\begin{solution}
It is easy to deduce the middle square must be $3$. Then the left and right squares must be one of ${1,2}$ and one of ${4,5}$ in some order, and then the top and bottom must be the remaining two in some order. This gives a total of $2 \cdot 2 \cdot 2 \cdot 2 = 16$ possibilities.
\end{solution}


\begin{problem}
If $x,y$ are distinct real numbers such that \[ x^2 + 4y = 25 \] \[y^2 + 4x = 25\] find $x+y$.
\end{problem}
\begin{answer}
4
\end{answer}
\begin{solution}
Subtracting the equations gives $x^2 - y^2 + 4y - 4x = 0$ or $(x-y)(x+y-4) = 0$. Since $x \ne y$, we see $x+y -4 = 0$ or $x+y = 4$. 
\end{solution}


\begin{problem}
Suppose the area of this L-shaped figure is $300$. Find the radius of the largest circle that can fit inside this figure. Express your answer in the form $a - b\sqrt{2}$.
\begin{figure}[H]
	\begin{center}
		\begin{asy}
		import graph;
		size(60);
		draw((0,0)--(0,2)--(1,2)--(1,1)--(2,1)--(2,0)--(0,0));
		draw(Circle((0.58578,0.58578),0.58578));
\end{asy}
	\end{center}
\end{figure}  
\end{problem}

\begin{answer}
$20-10\sqrt2$
\end{answer}
\begin{solution}
If the radius of the circle is $r$, notice the diagonal of the middle of the three squares is $r + \sqrt2 r$. The side length satisfies $3s^2 = 300$ so $s = 10$. Thus $r(1 + \sqrt2) = 10\sqrt2$ and $r = 10\sqrt2 (\sqrt2 - 1) = 20 - 10\sqrt{2}$. 
\end{solution}

\begin{problem}
In a regular tetrahedron $ABCD$ with volume $400$, we inscribe a sphere tangent to all four sides. Let $R$ be the center of the sphere. Find the volume of $[RABC]$.
\end{problem}
\begin{answer}
100
\end{answer}
\begin{solution}
Notice by symmetry $[RABC] = [RBCD] = [RACD] = [RABD]$ and they sum to $400$ so our answer is $100$.
\end{solution}

\end{document}