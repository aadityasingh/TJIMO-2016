\documentclass[11pt]{article}
\usepackage[paperwidth=8.5in, paperheight=11in]{geometry}

\usepackage{../tjimo}
%\usepackage[pdftex]{graphicx}
\usepackage{asymptote}

\begin{comment}
\def\answer{\comment}
\def\solution{\comment}
\def\solutionone{\comment}
\def\solutiontwo{\comment}
\end{comment}

\newcommand{\sevenpoints}{}
\newcommand{\righthead}{\fdbox{Round}{Practice Team}}

\begin{document}

\begin{problem}%[Alex Wang]
Determine the value of $(3^{3^2} - 3^{2^3} + 3^{2^2} - 2^{3^3})(3 + 3 - 3 + 2)^3(3^2 - 2^3 - 1)(3^3 + 2^2 + 1^1)$.
\end{problem}

\begin{answer}
$\boxed{0}$
\end{answer}

\begin{solution}
Since one of the factors is $3^2 - 2^3 - 1 = 9 - 8 - 1 = 0$, the entire product is simply $\boxed{0}$.
\end{solution}


\begin{problem}%[Lilian]
Michael has a playlist with $6$ songs in it, but $1$ of the songs is a repeat of another. If he presses shuffle, how many possible orders are there for his playlist?
\end{problem}

\begin{answer}
$\boxed{360}$ (orders)
\end{answer}

\begin{solution}
For $6$ songs, the number of ways to order them is $6! = 720$. However, since $2$ of them are the same, we overcounted by a factor of $2! = 2$, so $\frac{6!}{2!} = \frac{720}{2} = \boxed{360}$.
\end{solution}


\begin{problem}%[Bryan/Michael]
Edwin is swimming in a circular lake. He swims at $2\pi$ meters per minute. If it takes him $2400$ seconds to swim around the edge of the lake, what is the radius of the lake?
\end{problem}

\begin{answer}
$\boxed{40}$ (meters)
\end{answer}

\begin{solution}
Since $2400$ seconds is $\frac{2400}{60} = 40$ minutes, the circumference of the lake is $40 \cdot 2\pi = 80\pi$ meters. If the radius is $r$, then the circumference is given by $2\pi r$, so the radius is $r = \boxed{40}$ meters.
\end{solution}


\begin{problem}%[Ajit]
If $x + \frac{1}{x} = 4$, what is $x^2 + \frac{1}{x^2}$?
\end{problem}

\begin{answer}
$\boxed{14}$
\end{answer}

\begin{solution}
Observe that $\left(x + \frac{1}{x}\right)^2 = x^2 + 2x \cdot \frac{1}{x} + \frac{1}{x^2} = x^2 + \frac{1}{x^2} + 2$. Since $x + \frac{1}{x} = 4$, $\left(x + \frac{1}{x}\right)^2 = 16$, so $x^2 + \frac{1}{x^2} = 16 - 2 = \boxed{14}$.
\end{solution}


\begin{problem}
What is the area of a square with all four of its vertices on a circle of radius $10$?
\end{problem}

\begin{answer}
$\boxed{200}$ (square units)
\end{answer}

\begin{solution}
The four vertices must be evenly spaced around the circle, so the diagonals of the square are diameters of the circle, which have length $20$. A square is also a rhombus, so its area is $\frac{1}{2} \cdot 20 \cdot 20 = \boxed{200}$.
\end{solution}


\begin{problem}
Timmy rolls $4$ standard, fair, six-sided die. What is the probability that at least one of the number he rolls is prime?
\end{problem}

\begin{answer}
$\boxed{\frac{15}{16}}$
\end{answer}

\begin{solution}
The possible prime numbers that can be rolled are $2, 3$, and $5$, so there is a $\frac{3}{6} = \frac{1}{2}$ chance for each roll to be prime. Then there is a $1 - \frac{1}{2} = \frac{1}{2}$ probability for each roll to not be prime. Therefore, the probability that none of the $4$ dice show a prime number is $\left(\frac{1}{2}\right)^4 = \frac{1}{16}$, so the probability that at least one of them is prime is the complement, which is $1 - \frac{1}{16} = \boxed{\frac{15}{16}}$.
\end{solution}


\begin{problem}
Let $a, b$, and $c$ be real numbers such that $a+b+2c = 2015$, $a+2b+c = 2016$, and $2a+b+c=2017$. What is the value of $a+b+c$?
\end{problem}

\begin{answer}
$\boxed{1512}$
\end{answer}

\begin{solution}
Adding the three equations together gives us $4a + 4b + 4c = 2015 + 2016 + 2017 = 3 \cdot 2016$, so $a+b+c = 3 \cdot 504 = \boxed{1512}$.
\end{solution}

\begin{problem}%[George Tang, Daniel Mittal]
Aaditya is downloading an Android app. Every second, his phone has an equal chance to either download $20\%$ of the app or do nothing. What is the probability that after $8$ seconds, the app will have finished downloading?
\end{problem}

\begin{answer}
$\boxed{\frac{7}{32}}$
\end{answer}

\begin{solution}
There are $2^8 = 256$ possible sequences of events in $8$ seconds. In order for the app to have finished downloading after $8$ seconds, $5$ seconds must have been spent downloading $20\%$ of the app each, and $3$ seconds must have been spent doing nothing. There are $\binom{8}{5} = \frac{8 \cdot 7 \cdot 6}{3 \cdot 2} = 56$ different combinations of $5$ out of $8$ seconds for the phone to have spent downloading. Therefore the probability is $\frac{56}{256} = \boxed{\frac{7}{32}}$.
\end{solution}


\begin{problem}
A triangle with integer side lengths has a perimeter of $5$. What is its area?
\end{problem}

\begin{answer}
$\boxed{\frac{\sqrt{15}}{4}}$ (square units)
\end{answer}

\begin{solution}
By the triangle inequality, the only possible triangle has side lengths $2$, $2$, and $1$. This is an isosceles triangle with base $1$ and legs of length $2$. Its altitude is $\sqrt{2^2 - \left(\frac{1}{2}\right)^2} = \frac{\sqrt{15}}{2}$, so its area is $\frac{1}{2} \cdot 1 \cdot \frac{\sqrt{15}}{2} = \boxed{\frac{\sqrt{15}}{4}}$.
\end{solution}


\begin{problem}%[Josh/Eric Chen]
What is the largest prime factor of $17^3 + 1$?
\end{problem}

\begin{answer}
$\boxed{13}$
\end{answer}

\begin{solution}
Recall the sum of cubes factorization $a^3 + b^3 = (a+b)(a^2 - ab + b^2)$. Thus we have $17^3 + 1 = (17+1)(17^2 - 17\cdot 1 + 1^2) = 18(273) = (2 \cdot 3^2)(3 \cdot 7 \cdot 13)$, so its largest prime factor is $\boxed{13}$.
\end{solution}

\end{document}