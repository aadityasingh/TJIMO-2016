\documentclass[11pt]{article}
\usepackage[paperwidth=8.5in, paperheight=11in]{geometry}

\usepackage{../tjimo}
%\usepackage[pdftex]{graphicx}
\usepackage{asymptote}

\begin{comment}
\def\answer{\comment}
\def\solution{\comment}
\def\solutionone{\comment}
\def\solutiontwo{\comment}
\end{comment}

\newcommand{\sevenpoints}{Time limit: 60 minutes.}
\newcommand{\righthead}{\fdbox{Round}{Individual}}

\begin{document}


\begin{problem}
Compute $2^{0 + 1 \cdot 6} - 2^0 \cdot 1 + 7$.
\end{problem}

\begin{answer}
$\boxed{70}$
\end{answer}

\begin{solution}
We follow the order of operations to obtain \begin{align*} 2^{0 + 1 \cdot 6} - 2^0 \cdot 1 + 7 &= 2^{0 + 6} - 1 \cdot 1 + 7 \\ &= 2^6 - 1 + 7 \\ &= 64 - 1 + 7 = \boxed{70}. \end{align*}
\end{solution}


\begin{problem}
The sum of the digits of the year $2016$ is $2 + 0 + 1 + 6 = 9$. Find the next year for which the sum of the digits is also $9$.
\end{problem}

\begin{answer}
\boxed{2025}
\end{answer}

\begin{solution}
Note that for $2017, 2018$, and $2019$ have sum of digits greater than $9$. The sum of digits of $2020$ is $4$, so we need $9-4 = 5$ more years. Hence $\boxed{2025}$ is the next year with sum of digits equal to $9$.
\end{solution}


\begin{problem}
Compute $1 + 2 + 3 + \cdots +  63$. 
\end{problem}

\begin{answer}
$\boxed{2016}$
\end{answer}

\begin{solution}
Note that $1 + 63 = 64$, $2 + 62 = 64$, $3 + 61 = 64$, and so forth. Therefore $(1 + 63) + (2 + 62) + (3 + 61) + \cdots (63 + 1) = 63 \cdot 64$ is twice the desired sum. Thus the desired sum is $\frac{1}{2} \cdot 63 \cdot 64 = 63 \cdot 32 = \boxed{2016}$.
\end{solution}


\begin{problem}%[Kyle]
A certain circle has an area whose value is twice the value of its circumference. Compute the diameter of the circle.
\end{problem}

\begin{answer}
$\boxed{8}$
\end{answer}

\begin{solution}
The area of a circle with radius $r$ is $\pi r^2$, and its circumference is $2\pi r$. We have that $\pi r^2 = 2 \cdot 2\pi r$. Dividing by $\pi r$ on both sides leaves us with $r = 4$, so the diameter is $\boxed{8}$.
\end{solution}


\begin{problem}
Polya has 5 shirts, each of a different color: red, green, blue, yellow, and black. He also has 4 hats of different colors: red, orange, yellow, green. Compute the number of ways he can choose a shirt and a hat.
\end{problem}

\begin{answer}
$\boxed{20}$ (ways)
\end{answer}

\begin{solution}
There are $5$ ways to choose a shirt. For each shirt, there are $4$ ways to choose a hat. Thus there are $5 \cdot 4 = \boxed{20}$ total ways. 
\end{solution}


\begin{problem}
Cantor normally takes $30$ minutes to walk to school at $3$ miles per hour. One day, he left home $10$ minutes later than usual. Compute the speed, in miles per hour, at which he must travel to still get to school on time. Express your answer as a decimal to the nearest tenth.
\end{problem}

\begin{answer}
$\boxed{4.5}$ (miles per hour)
\end{answer}

\begin{solution}
The distance from Cantor's home to school is $\frac{1}{2} \text{hour} \cdot 3 \text{mph} = 1.5$ miles. If he leaves home $10$ minutes late, then he needs travel the same distance in $20$ minutes, or $\frac{1}{3}$ hour. Therefore he must travel at $\frac{1.5 \text{ miles}}{\frac{1}{3} \text{ hour}} = \boxed{4.5}$ mph.
\end{solution}


\begin{problem}
Compute the number of positive divisors of $1024$.
\end{problem}

\begin{answer}
$\boxed{11}$
\end{answer}

\begin{solution}
The prime factorization of $1024$ is $2^{10}$. Thus, any divisor of $1024$ must be a power of $2$. We can represent any divisor as $2^x$ for some $0 \le x \le 10$. For example, $x=0$ gives $1$ and $x=10$ gives $1024$. Since there are $11$ possible values for $x$, there are then $\boxed{11}$ positive divisors of $1024$. 
\end{solution}


\begin{problem} %[NO NAME]
Fermat was selling his very last and little mat at an original price of $\$20$. He then decided to give a discount at $10\%$ off of the original price. After 2 weeks, no one bought his mat. Furious, he decided to give an additional $30\%$ off the discounted price. Finally, after both discounts, Andrew Wiles bought the mat. Find the price Andrew Wiles paid. Express your answer in dollars and cents.
\end{problem}

\begin{answer}
$\boxed{\$12.60}$
\end{answer}

\begin{solution}
If the original price is $\$20$, the first discount takes off $10\% \cdot \$20  = \$2$, resulting in a price of $\$20 - \$2 = \$18$. Then the next discount takes off $30\%$ of $\$18$, or $0.30 \cdot \$18 = \$5.40$, resulting in a price of $\$18 - \$5.40 = \boxed{\$12.60}$, which is the price Andrew Wiles must have paid. 
\end{solution}


\begin{problem}
%logic/puzzle question -not sure if this counts?
%If the midterm and final exams of a TJ chemistry class are each worth $\frac{1}{9}$ of the overall grade, and each quarter grade is worth $\frac{7}{36}$ of the overall grade, and NAME gets an average of 91 across all four of his quarter grades, and gets a 73 on his midterm, what score must he receive on the final exam on the final exam in order to get an A (overall grade $\geq 92.5$)?

Al, Bob, Carl, David, and Edward are standing side by side for a picture. Carl is to the left of Al, Bob is to the right of Edward, and David is to the left of Edward. Furthermore, David is complaining that the people on both of his sides are not giving him enough space. Find the person standing farthest to the left. (Note: ``Carl is to the left of Al" does not necessarily mean Carl is adjacent to Al.)
%  AD  BCE    BADCE           DEB   CA
\end{problem}

\begin{answer}
%$122.5$
$\boxed{\text{Carl}}$
\end{answer}

\begin{solution}
We know that the order for Al and Carl is Carl, Al, and the order for Bob, David, and Edward is David, Edward, Bob. Then neither Al, Edward, nor Bob can be farthest to the left. Furthermore, David is between two people, so $\boxed{\text{Carl}}$ must be the farthest to the left. 
\end{solution}

\begin{problem}
Mario has a humongous jug with $640$ ounces of apple juice. Every day, Mario drinks half of the contents in the jug. Thus, after the first day, there will be $320$ ounces remaining. Compute the number of days after which there will be $5$ ounces of apple juice left.
\end{problem}

\begin{answer}
$\boxed{7}$ (days)
\end{answer}

\begin{solution}
Every day, Mario divides the amount of juice left by $2$. To reach $5$ ounces, he must divide the original amount by $\dfrac{640}{5} = 128$. Since $128 = 2^7$, Mario must drink from the jug seven times, so it will take him $\boxed{7}$ days. 
\end{solution}

\begin{problem}
Define the operation $a \triangle b = a + 2b$ for any real $a, b$. If $a \triangle (b \triangle a)  = (ka) \triangle b$ for all positive integers $a$ and $b$, compute $k$.
\end{problem}

\begin{answer}
$\boxed{5}$
\end{answer}

\begin{solution}
We simply evaluate  both sides of the equation: \begin{align*}a \triangle (b \triangle a) = a + 2(b \triangle a) &= a + 2(b + 2a) = 5a + 2b \\ (ka) \triangle b &= ka + 2b\end{align*} Thus we must have $5a = ka$ for all positive integers $a$, so $k = 5$.
\end{solution}

\begin{problem}
At a vending machine, 3 bags of chips and 2 bottles of water cost $\$2.35$, and 2 bags of chips and 3 bottles of water cost $\$2.65$. Compute the cost of 1 bag of chips and 1 bottle of water.
\end{problem}

\begin{answer}
$\boxed{\$1}$
\end{answer}

\begin{solution}
Let $c$ be the cost of one bag of chips, and $w$ be the cost of one bottle of water. We have: \begin{center}$\begin{cases} 3c + 2w = 2.35 \\ 2c + 3w = 2.65 \end{cases}$\end{center} Adding these equations together gives $5c + 5w = 5$, so $c + w = \boxed{1}$. (Although not necessary, we can also solve for $w$ and $c$ explicitly to find $w = 0.35$ and $c = 0.65$.)
\end{solution}

\begin{problem}
How many unordered sets of three prime numbers sum to $38$? (Note: the set \{1, 2, 3\} is considered the same as the set \{3, 2, 1\})
\end{problem}

\begin{answer}
$\boxed{4}$
\end{answer}

\begin{solution}
First notice that if all three prime numbers were odd, then their sum would be odd. Thus, at least one must be even. So, one of the three primes is $2$. The other two primes must sum to $36$. It is easy to check that the only pairs of primes that sum to $36$ are $\{5, 31\}, \{7, 29\}, \{13, 23\}, \{17, 19\}$. Thus, there are $\boxed{4}$ total sets.
\end{solution}

\begin{problem}
A palindrome is a positive integer that reads the same forwards and backwards. For example, $2002$ and $1111$ are palindromes, while $2017$ is not. Compute the number of $5$-digit palindromes.
\end{problem}

\begin{answer}
$\boxed{900}$
\end{answer}

\begin{solution}
Note that a five-digit palindrome is of the form $\overline{ABCBA}$, where $A,B,C$ are digits and $A \ne 0$. There are $9$ ways to choose $A$, and $10$ ways to choose each of $B$ and $C$, so our answer is $9 \cdot 10 \cdot 10 = \boxed{900}$.
\end{solution}

\begin{problem}
Determine the greatest integer from the following set: $\{7^{2000},\  2^{6000},\ 7^{1000} \times 3^{2000},\ 3^{1000} \times 5^{2000},\ 3^{4000} \}$
\end{problem}

\begin{answer}
$\boxed{3^{4000}}$
\end{answer}

\begin{solution}
Take the $1000$th root of each number. Note that this does not affect the largest number. We then obtain the set $\{7^2, 2^6, 7 \cdot 3^2, 3 \cdot 5^2, 3^4\}$, or $\{49, 64, 63, 75, 81\}$. The largest number is $81$, so our answer is $\boxed{3^{4000}}$.
\end{solution}

\begin{problem}
Find $a$ such that the following system of equations have no solutions: $\begin{cases} 2x + 3y = 7 \\ 5x + ay = 10 \end{cases}$
\end{problem}

\begin{answer}
$\boxed{\dfrac{15}{2}}$
\end{answer}

\begin{solution}
Notice multiplying the first equation by $\dfrac{5}{2}$ gives $5x + \dfrac{15}{2}y = \dfrac{35}{2}$. For the system to have no solutions notice we can make $a = \boxed{\dfrac{15}{2}}$ to make the system inconsistent (since $10 \ne \frac{35}{2}$). 
\end{solution}

%\begin{problem}
%Seven people named Affery, Beffery, Ceffery, Deffery, Effery, Feffery, and Geffery want to form a committee. How many possible committees are there that contain both Affery and Beffery but not Ceffery?
%\end{problem}
%
%\begin{answer}
%$\boxed{16}$
%\end{answer}
%
%\begin{solution}
%The committee already must contain Affery and Beffery and not Ceffery. For the four other people, each of them has two options: either join the committee or not. Thus, in total there are $2^4 = \boxed{16}$ possibilities. 
%\end{solution}


\begin{problem}
Lewis has $25$ mL of a $20\%$ acid solution and wants to make a $40\%$ acid solution. However, he accidentally adds $10$ mL of additional pure water before realizing he was supposed to add acid. Determine the amount of pure acid (in mL) he now needs to add in order to make his $40\%$ acid solution.
\end{problem}

\begin{answer}
$\boxed{15}$ (mL)
\end{answer}

\begin{solution}
The original solution consisted of $20\% \cdot 25 = 5$ mL of acid and $20$ mL water. The additional $10$ mL of water gives a total of $30$ mL of water. Since only pure acid is added, the amount of water remains constant. Thus the final $40\%$ acid solution will contain $30$ mL of water, which forms $60\%$ of the solution. Therefore $40\%$ of the solution will be $20$ mL acid, so $20 - 5 = \boxed{15}$ mL of pure acid needs to be added.
\end{solution}


\begin{problem}%[Joie]
Find the perimeter of the figure below, where $AB = 6$, $AJ = 10$, $GH = 1$, and all interior angles either measure $90^\circ$ or $270^\circ$.
\begin{figure}[h]
	\begin{center}
		\begin{asy}
		unitsize(10);
		pair A = (0, 0), B = (0, 6), C = (4, 6), D = (4, 4), E= (6, 4), F = (6, 5), G = (8, 5), H = (8, 4), I = (10, 4), J = (10, 0);
		draw(A--B--C--D--E--F--G--H--I--J--cycle);
		label("A", A, S);
		label("B", B, N);
		label("C", C, N);
		label("D", D, S);
		label("E", E, S);
		label("F", F, N);
		label("G", G, N);
		label("H", H, S);
		label("I", I, N);
		label("J", J, S);
		\end{asy}
	\end{center}
\end{figure}
\end{problem}

\begin{answer}
$\boxed{34}$
\end{answer}

\begin{solution}
The key realization here is that $BC+DE+FG+HI = AJ = 10$ and $CD+IJ = AB = 6$. Thus, the perimeter is $2(10+6) + EF+GH=32 + 1 + 1 = \boxed{34}$.
\end{solution}


\begin{problem}
Compute the units digit of $2017^{2016}$.
\end{problem}

\begin{answer}
$\boxed{1}$
\end{answer}

\begin{solution}
We may start by simply replacing $2017$ with its units digit $7$. By looking at the first few powers of $7$ (7, 49, 343, 2401, 16807), we see that the units digit cycles every $4$ powers in the pattern $7, 9, 3, 1, 7, 9, 3, 1, \ldots$. Since every fourth power has units digit $1$ and $2016$ is divisible by $4$, our answer is $\boxed{1}$.
\end{solution}


\begin{problem}
Two standard six-sided dice are rolled. Compute the probability that the product of the numbers rolled is divisible by $2$ or $3$. Express your answer as a common fraction.
\end{problem}

\begin{answer} % Fixed problem statmeent i think
$\boxed{\frac{8}{9}}$
\end{answer}

\begin{solution}
We will use complementary counting. That is, we compute the probability that the product of the numbers rolled is not divisible by $2$ or $3$ and subtract that from $1$. In order to not be divisible by $2$ or $3$, each die can either show a $1$ or $5$, for a $\frac{2}{6} = \frac{1}{3}$ chance each. For two independent dice, the probability is $\frac{1}{3} \cdot \frac{1}{3}$. Therefore the probability that the product is divisible by $2$ or $3$ is $1 - \frac{1}{9} = \boxed{\frac{8}{9}}$.
\end{solution}


\begin{problem}
Zermelo, Zorn, and Zsigmondy are in a race. If Zermelo always beats Zsigmondy, but otherwise ties are allowed, find the number of possible results. For example, Zorn and Zermelo at a tie for first and Zsigmondy second is one such result.
\end{problem}

\begin{answer}
$\boxed{5}$
\end{answer}

\begin{solution}
Suppose only Zermelo is first. Then Zorn can beat, tie, or lose to Zsigmondy, for $3$ possibilities. If only Zorn is first, then Zermelo must be second and Zsigmondy third. Finally, if Zermelo and Zorn are tied for first, then Zsigmondy must be second. This gives a total of $\boxed{5}$ possible results.
\end{solution}


\begin{problem}
Dr. Os's class can normally be split evenly into equal groups of $7$ when everyone is present. When $2$ people are absent and they try to split into groups of $9$, they are $1$ person short of having equal groups. Find the minimum number of students in the entire class.
\end{problem}

\begin{answer}
$\boxed{28}$ (students)
\end{answer}

\begin{solution}
If there are $n$ students in the class, then we know that $n$ is divisible by $7$ and $n-1$ is divisible by $9$, or $n$ is $1$ more than a multiple of $9$. Simply listing out multiples of $7$, we have $7, 14, 21, 28, \ldots$. We see that $\boxed{28}$ is the first multiple of $7$ that is $1$ more than a multiple of $9$, so that is our answer.
\end{solution}


\begin{problem}%[Michael You]
Rectangle $CDEF$ is inscribed in right triangle $ABC$ with right angle $C$ such that $D$ is on $AC$, $E$ is on $AB$, and $F$ is on $BC$. If $AD=3, CD=4$, and $DE=5$, find the area of $\triangle ABC$. Express your answer as a common fraction.
%ADD ASYMPTOTE 
\end{problem}

\begin{answer}
$\boxed{\frac{245}{6}}$
\end{answer}

\begin{solution}
Since $\triangle AED \sim \triangle ABC$, we see that $BC = \frac{ED}{AD} \cdot AC = \frac{5}{3} \cdot 7 = \frac{35}{3}$. Thus, the area is $\dfrac{7 \cdot \frac{35}{3}}{2} = \boxed{\frac{245}{6}}$
\end{solution}

\begin{problem}%[Lilian]
If $\frac{1}{x} + \frac{1}{y} = 5$ and $x+y=10$, find the value of $x^2+y^2$.
\end{problem}

\begin{answer}
$\boxed{96}$
\end{answer}

\begin{solution}
We can rewrite the first equation as $\dfrac{x+y}{xy} = 5$. Thus, since $x+y = 10$, we know $xy = 2$. Now notice $x^2+y^2 = (x+y)^2 - 2xy = 10^2 - 2 \cdot 2 = \boxed{96}$.
\end{solution}

\begin{problem}%[Kyle]
Two tanks are used to collect water to drain a full swimming pool. When only tank A is opened, it takes $2$ hours for the pool to drain completely. When only tank B is opened, it takes $3$ hours for the pool to drain completely. Compute how many \textbf{minutes} it will take to drain the pool if both tanks are opened from the start.
\end{problem}

\begin{answer}
$\boxed{72}$ (minutes)
\end{answer}

\begin{solution}
We describe tanks A and B in terms of their individual drainage rates. Tank A has a rate of 1 swimming pool/2 hours. Tank B has a rate of 1 swimming pool/3 hours. Thus, the combined rate of tanks A and B working together is $\frac{1}{2} + \frac{1}{3} = \frac{5}{6}$, which can be rewritten in terms of our full pool: $\frac{5}{6} =$ 1 swimming pool/$\frac{6}{5}$ hours. hence it takes $\frac{6}{5}$ hours, or $\boxed{72}$ minutes, for both tanks to drain the pool.
\end{solution}


\begin{problem}%[Kyle/Fred]
Compute the smallest positive integer value of $n$ such that $n!$ ($n$ factorial) ends in $16$ consecutive zeros, where $n! = 1 \cdot 2 \cdot \ldots \cdot n$.
\end{problem}

\begin{answer}
$\boxed{70}$
\end{answer}

\begin{solution}
The number of zeros at the end of a number $n$ is equal to the largest power $k$ such that $10^k$ is a factor of $n$. We solve this by incrementing $n$ until $n$ acquires all $16$ zeros. We will always multiply by a multiple of $2$ more frequently than a multiple of $5$, so every time we multiply our current factorial by a multiple of $5^p$, we increase the number of zeros by $p$. The numbers $5, 10, 15$, and $20$ each add one zero. Next, $25$ adds two zeros, so $25!$ has 6 terminal zeros. We add another $6$ zeros to the end by the time we reach $50!$, with $12$ terminal zeros. We add four more zeros to reach $16$ with the numbers $55, 60, 65$, and $70$. Hence, $70!$ is the smallest factorial ending in $16$ zeros, so $n = \boxed{70}$.
\end{solution}


\begin{problem}%[Michael You]
Two cars are traveling towards each other, with car A traveling at 70 km/h and car B traveling at 65 km/hr. Initially, they are 15 km apart, and a bird begins to fly from car A to car B. When the bird reaches car B, it immediately starts flying back to car A. The bird continues to do this until the cars collide. Given that the bird flies at a speed of 20 km/h, compute the total distance the bird will fly before stopping. Express your answer as a common fraction.
\end{problem}

\begin{answer}
$\boxed{\frac{20}{9}}$ (km)
\end{answer}

\begin{solution}
The amount of time the bird flies for is the same amount of time it takes for the cars to collide. The cars will collide in $\frac{15}{70+65}= \frac{1}{9}$ h. Thus, the distance is $20 \cdot \frac{1}{9} = \boxed{\frac{20}{9}}$
\end{solution}


\begin{problem}%[Kyle]
Two circles with equal radii are placed inside a $16 \times 18$ rectangular box as shown in the diagram. Each circle touches two walls of the box. The two circles are also tangent to each other at one point inside the box. Compute the radius of the circles.
\begin{figure}[h]
    \begin{center}
        \begin{asy}
        import graph;
        unitsize(3);
        draw((0, 0) -- (16, 0) -- (16, 18) -- (0, 18) -- cycle);
        draw(Circle((5, 5), 5));
        draw(Circle((11, 13), 5));
        label("16", (8, 0), S);
        label("18", (0, 9), W);
        \end{asy}
    \end{center}
\end{figure}
\end{problem}

\begin{answer}
$\boxed{5}$
\end{answer}

\begin{solution}
The point of tangency of the circles is the center of the rectangle. Therefore, we can compress the original rectangle by a factor of $\frac{1}{2}$ to make one of its corners lie on the tangency point of the two circles, as shown. Then we add in line segments representing the radii of one of our circles, each radius directed toward a different point of tangency.

\begin{figure}[h]
    \begin{center}
        \begin{asy}
        import graph;
        unitsize(4);
        draw((0, 0) -- (16, 0) -- (16, 18) -- (0, 18) -- cycle);
        draw((5, 5) -- (8, 9));
        draw((5, 5) -- (5, 0));
        draw((5, 5) -- (0, 5));
        draw((0, 9) -- (8, 9) -- (8, 0));
        filldraw((5, 5) -- (8, 5) -- (8, 9) -- cycle, grey);
        draw(Circle((5, 5), 5));
        draw(Circle((11, 13), 5));
        label("8", (4, 0), S);
        label("9", (0, 4.5), W);
        \end{asy}
    \end{center}
\end{figure} 

Now we consider the shaded right triangle in the diagram. If we let $r$ be the radius of the circle, the right triangle has legs of length $8-r$ and $9-r$ and hypotenuse of length $r$. Thus, by the Pythagorean Theorem, we have $(8-r)^2 + (9-r)^2 = r^2$, which reduces to $r^2 - 34r + 145 = 0$. Factoring this gives us $(r-5)(r-29) = 0$. The radius of the circle cannot be $29$ (the circle needs to fit inside a $16 \times 18$ box), so our only solution is $r = \boxed{5}$.
\end{solution}

\begin{problem}
Find the least positive integer whose sum of positive divisors is $42$.
\end{problem}

\begin{answer}
$\boxed{20}$
\end{answer}

\begin{solution}
For a general number whose prime factorization is $p_1^{e_1}p_2^{e_2} \cdots p_k^{e_k}$, the sum of its divisors is $(1 + p_1 + p_1^2 \cdots + p_1^{e_1}) \cdots (1 + p_k + \cdots + p_k^{e_k})$. We have $42 = 2 \cdot 3 \cdot 7$, but $2$ is not workable ($1+1$ is invalid as $1$ is not prime), so the possible groupings are $6 \cdot 7 = (1 + 5)(1 + 2 + 4)$, $3 \cdot 14 = (1 + 2)(1 + 13) $, and $42 = 1 + 41$. Of these, the least option is $(1+5)(1+2+4)$, which corresponds to the number $5 \cdot 4 = \boxed{20}$.
\end{solution}


\begin{problem}
Aaditya has a random number generator that randomly selects a real number $x$ uniformly at random between $0$ and $1$. He uses his random number generator twice to obtain two real numbers $a$ and $b$. What is the probability that $\dfrac{1}{6} < \dfrac{a}{a+b} < \dfrac{5}{6}$?
\end{problem}

\begin{answer}
$\boxed{\frac{4}{5}}$
\end{answer}

\begin{solution}
Since the random number generator has a continuous distribution we model this with geometric probability on a coordinate plane. Let the $x$-axis represent the value of $a$ and the $y$-axis that of $b$. Cross multiplying $\dfrac{1}{6} < \dfrac{x}{x+y}$ gives $x+y < 6x$, or $y < 5x$. Similarly, $\dfrac{x}{x+y} < \dfrac{5}{6}$ requires $6x < 5x+5y$, or $x < 5y$.

\begin{figure}[H]
	\begin{center}
		\begin{asy}
		import graph;
		pair A = (0, 0), B = (0.2, 1), C = (1, 0.2);
		unitsize(60);
		xaxis("$x$", -0.1, 1.5);
		yaxis("$y$",-0.1, 1.5);
		real f(real x) {return 5*x;}
		real g(real x) {return x/5;}

		draw((0, 0) -- (1, 0) -- (1, 1) -- (0, 1) -- cycle);
		draw(graph(f, 0, 0.3, operator ..));
		draw(graph(g, 0, 1.3, operator ..));
		filldraw(A -- B -- (1, 1) -- C -- cycle, gray);
		label("1", (1, 0), S);
		label("1", (0, 1), W);
		\end{asy}
	\end{center}
\end{figure}

Now we can graph both inequalities in the Cartesian plane. Note that our answer is simply the area of the region within the square $0 < x < 1$ and $0 < y < 1$ between the lines $y=5x$ and $y = \dfrac{1}{5}x$, divided by the area of the square, which is $1$. To find the successful area, we note that the complement area is two congruent triangles both which have area $\dfrac{1}{2} \cdot \dfrac{1}{5} \cdot 1  = \dfrac{1}{10}$, so our answer is $1 - 2 \cdot \dfrac{1}{10} = \boxed{\dfrac{4}{5}}$.
\end{solution}
\end{document}